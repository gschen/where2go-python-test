\documentclass[12pt]{article}
\usepackage[a4paper, left=0.5in, right=0.5in, top=0.5in, bottom=0.5in]{geometry}
\usepackage{xeCJK}
\usepackage{graphicx}
\usepackage{float}
\usepackage{setspace}
\usepackage{cite}
\usepackage{amsmath}
\usepackage{subfigure}
\usepackage{longtable}


\usepackage{algorithmicx}
\usepackage[ruled]{algorithm}
\usepackage{algpseudocode}

\usepackage{tikz}
\usetikzlibrary{shapes.geometric, arrows}

\setmainfont{Times New Roman}  
\setCJKmainfont{Songti SC}

\title{Latex示例教程}
\author{张三}
\date{\today}

\begin{document}
\doublespacing
\maketitle

\section{为什么要用Latex写作}

\section{准备工作}

\section{基本用法}
\subsection{数学公式}
$ 1+2+\cdots+100=\frac{1+100}{2}*100$

\subsection{列表}

\begin{itemize}
    \item 列表1
    \item 列表2
    \item ...
\end{itemize}

\subsection{插入图片}
插入图片,图片自动编号。
\begin{figure}[H]
    \centering
    \includegraphics[scale=0.5]{logo.jpg}
    \caption{这是一张图片}
    \label{}
\end{figure}

\subsection{算法伪码}

\begin{algorithm}[H]
    \caption{判断像素点是否为火焰}\label{ga-algo}
    \begin{algorithmic}[1]
        \Require aa
        \Ensure bb
        \Procedure{is\_fire\_pixel}{$ a,b$}
        \Comment{判断给定的像素点是不是火焰像素点}
        \State rule1: R > G > B
        \State rule2: 
        \State rule3: 
        \If {rule1}
        \State return fire
        \EndIf 
        \EndProcedure
    \end{algorithmic}
\end{algorithm}

\subsection{流程图}

% 定义流程图中各种组件样式
\tikzstyle{startstop} = [rectangle, rounded corners, minimum width=3cm, minimum height=1cm,text centered, draw=black, fill=red!30]
\tikzstyle{io} = [trapezium, trapezium left angle=70, trapezium right angle=110, minimum width=3cm, minimum height=1cm, text centered, draw=black, fill=blue!30]
\tikzstyle{process} = [rectangle, minimum width=3cm, minimum height=1cm, text centered, text width=3cm, draw=black, fill=orange!30]
\tikzstyle{decision} = [diamond, minimum width=3cm, minimum height=1cm, text centered, draw=black, fill=green!30]
\tikzstyle{arrow} = [thick,->,>=stealth]


\begin{figure}[H]
    \centering
    \begin{tikzpicture}[node distance=2cm]
        % 第一步画结点
        \node (start) [startstop] {开始};
        \node (in1) [io, below of=start] {Input};
        \node (pro1) [process, below of=in1] {Process 1};
        \node (dec1) [decision, below of=pro1, yshift=-0.5cm] {Decision 1};
        \node (pro2a) [process, below of=dec1, yshift=-1.5cm] {Process 2a long long long text long long long text};
        \node (pro2b) [process, right of=dec1, xshift=2cm] {Process 2b};
        \node (pro3a)[process, left of=pro2a, yshift=-4cm, xshift=-2cm]{Process 3a};
        \node (pro3b)[process, below of=pro2a, yshift=-2cm]{Process 3b};
        \node (pro3c)[process, right of=pro2a,  yshift=-4cm, xshift=2cm]{Process 3c};
        \node (out1) [io, below of=pro3b, yshift=-1.5cm] {Output};
        \node (stop) [startstop, below of=out1] {Stop};

        % 第二步画线
        \draw [arrow] (start) -- (in1);
        \draw [arrow] (in1) -- (pro1);
        \draw [arrow] (pro1) -- (dec1);
        \draw [arrow] (dec1) -- node[anchor=east]{yes} (pro2a);
        \draw [arrow] (dec1) -- node[anchor=south]{no} (pro2b);
        \draw [arrow] (pro2b) |- (pro1); %折线画法
        \draw [arrow] (pro2a) -| (pro3a);
        \draw [arrow] (pro2a) -- (pro3b);
        \draw [arrow] (pro2a) -| (pro3c);
        \draw [arrow] (pro3a) |- (out1);
        \draw [arrow] (pro3b) -- (out1);
        \draw [arrow] (pro3c) |- (out1);
        \draw [arrow] (out1) -- (stop);
    \end{tikzpicture}
    \caption{算法流程图}
\end{figure}

%定义形状样式
\tikzstyle{startstop} = [rectangle, rounded corners, minimum width = 3cm, minimum height = 0.7cm, text centered, draw = black]
\tikzstyle{startstop2} = [rectangle, rounded corners, minimum width = 13cm, minimum height = 0.7cm, text centered, draw = black]
\tikzstyle{io} = [trapezium, trapezium left angle = 30, trapezium right angle = 150, minimum width = 3cm, text centered, draw = black, fill = white]
\tikzstyle{io2} = [trapezium, trapezium left angle = 30, trapezium right angle = 150, minimum width = 2.5cm, draw = black, fill = white]
\tikzstyle{io3} = [trapezium, trapezium left angle = 30, trapezium right angle = 150, minimum width = 2cm, draw = black, fill = white]
\tikzstyle{process} = [rectangle, minimum width = 3cm, minimum height = 1cm, text centered, draw = black]
\tikzstyle{decision} = [diamond, minimum width = 3cm, minimum height = 1cm, text centered, draw = black]
\tikzstyle{arrow} = [thick, -, >= stealth]
\tikzstyle{arrow2} = [thick, ->, >= stealth]

\begin{figure}[H]
    \centering
    \begin{tikzpicture}[node distance = 1.5cm]
    % 定义流程图具体形状
    \coordinate[label = left:{\small 输入图像}](A) at(-1.5, 0);
    \node(in1) [io] {};
    \node(pro1) [startstop, below of = in1] {\small 线性滤波};
    
    \node(in2 - 2)[io3, below of = pro1, yshift = -0.6cm]{};
    \node(in3 - 2)[io3, left of = in2 - 2, xshift = -2.5cm]{};
    \node(in4 - 2)[io3, right of = in2 - 2, xshift = 2.5cm]{};
    
    \node(in2 - 1)[io2, below of = pro1, yshift = -0.3cm]{};
    \node(in3 - 1)[io2, left of = in2 - 1, xshift = -2.5cm]{};
    \node(in4 - 1)[io2, right of = in2 - 1, xshift = 2.5cm]{};
    
    \node(in2) [io, below of = pro1] {\small 颜色};
    \node(in3)[io, left of = in2, xshift = -2.5cm]{ \small 亮度 };
    \node(in4)[io, right of = in2, xshift = 2.5cm]{ \small 方向 };
    
    \node(in5)[startstop2, below of = in2 - 2]{ \small Center - Surround差异计算及归一化 };
    
    \node(in6 - 2)[io3, below of = in5, yshift = -0.6cm]{};
    \node(in7 - 2)[io3, left of = in6 - 2, xshift = -2.5cm]{};
    \node(in8 - 2)[io3, right of = in6 - 2, xshift = 2.5cm]{};
    
    \node(in6 - 1)[io2, below of = in5, yshift = -0.3cm]{};
    \node(in7 - 1)[io2, left of = in6 - 1, xshift = -2.5cm]{};
    \node(in8 - 1)[io2, right of = in6 - 1, xshift = 2.5cm]{};
    
    \node(in6) [io, below of = in5] {};
    \node(in7)[io, left of = in6, xshift = -2.5cm]{};
    \node(in8)[io, right of = in6, xshift = 2.5cm]{};
    
    \coordinate[label = left:{\small 特征图}](B) at(-1, -6.2);
    \coordinate[label = left:{\small (12张)}](C) at(-1.5, -7.5);
    \coordinate[label = left:{\small (6张)}](D) at(2.7, -7.5);
    \coordinate[label = left:{\small (24张)}](E) at(6.7, -7.5);
    
    \node(in9)[startstop2, below of = in6 - 2]{ \small 跨尺度合并及归一化 };
    
    \node(in10) [io, below of = in9] {};
    \node(in11)[io, left of = in10, xshift = -2.5cm]{};
    \node(in12)[io, right of = in10, xshift = 2.5cm]{};
    
    \coordinate[label = left:{\small 醒目图}](F) at(-1, -9.5);
    \node(in13) [startstop, below of = in10] {\small 线性组合};
    \node(in14) [io, below of = in13] {};
    \coordinate[label = left:{\small 显著图}](G) at(-1, -13);
    
    \node(in15) [startstop, below of = in14] {\small 赢者取全};
    \coordinate[label = left:{\small 显著位置}]() at(1, -16.1);
    \coordinate[label = left:{\small 反馈抑制}]() at(4.5, -14.7);
    
    %连线
    \draw[arrow](pro1) -- (in1);
    \draw[arrow](pro1) -- (in2);
    \draw[arrow](pro1) -- (in3);
    \draw[arrow](pro1) -- (in4);
    \draw[arrow](0, -4.75) -- (in2 - 2);
    \draw[arrow](-4, -4.75) -- (in3 - 2);
    \draw[arrow](4, -4.75) -- (in4 - 2);
    \draw[arrow](0, -5.45) -- (in6);
    \draw[arrow](-4, -5.45) -- (in7);
    \draw[arrow](4, -5.45) -- (in8);
    \draw[arrow](0, -8.35) -- (in6 - 2);
    \draw[arrow](-4, -8.35) -- (in7 - 2);
    \draw[arrow](4, -8.35) -- (in8 - 2);
    \draw[arrow](0, -9.05) -- (in10);
    \draw[arrow](-4, -9.05) -- (in11);
    \draw[arrow](4, -9.05) -- (in12);
    \draw[arrow](in13) -- (in10);
    \draw[arrow](in13) -- (in11);
    \draw[arrow](in13) -- (in12);
    \draw[arrow](in13) -- (in14);
    \draw[arrow](in14) -- (in15);
    \draw[arrow](in15) -- (0, -15.8);
    \draw[arrow](0, -15.4) -- (2.5, -15.4);
    \draw[arrow](2.5, -14) -- (2.5, -15.4);
    \draw[arrow2](2.5, -14) -- (0, -14);
    \end{tikzpicture}
    \caption{ IT算法流程 }
\end{figure}
    
\section{结语}


\end{document}