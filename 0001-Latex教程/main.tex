\documentclass[12pt]{article}
\usepackage[a4paper, left=0.5in, right=0.5in, top=0.5in, bottom=0.5in]{geometry}
\usepackage{xeCJK}
\usepackage{graphicx}
\usepackage{float}
\usepackage{setspace}
\usepackage{cite}
\usepackage{amsmath}
\usepackage{subfigure}
\usepackage{longtable}


\usepackage{algorithmicx}
\usepackage[ruled]{algorithm}
\usepackage{algpseudocode}

\usepackage{tikz}
\usetikzlibrary{shapes.geometric, arrows}

\setmainfont{Times New Roman}  
\setCJKmainfont{Songti SC}

\title{Latex示例教程}
\author{张三}
\date{\today}

\begin{document}
\doublespacing
\maketitle

\section{为什么要用Latex写作}

\section{准备工作}

\section{基本用法}
\subsection{数学公式}
$ 1+2+\cdots+100=\frac{1+100}{2}*100$

\subsection{列表}

\begin{itemize}
    \item 列表1
    \item 列表2
    \item ...
\end{itemize}

\subsection{插入图片}
插入图片,图片自动编号。
\begin{figure}[H]
    \centering
    \includegraphics[scale=0.5]{logo.jpg}
    \caption{这是一张图片}
    \label{}
\end{figure}

\subsection{算法伪码}

\begin{algorithm}[H]
    \caption{判断像素点是否为火焰}\label{ga-algo}
    \begin{algorithmic}[1]
        \Require aa
        \Ensure bb
        \Procedure{is\_fire\_pixel}{$ a,b$}
        \Comment{判断给定的像素点是不是火焰像素点}
        \State rule1: R > G > B
        \State rule2: 
        \State rule3: 
        \If {rule1}
        \State return fire
        \EndIf 
        \EndProcedure
    \end{algorithmic}
\end{algorithm}

\subsection{流程图}

% 定义流程图中各种组件样式
\tikzstyle{startstop} = [rectangle, rounded corners, minimum width=3cm, minimum height=1cm,text centered, draw=black, fill=red!30]
\tikzstyle{io} = [trapezium, trapezium left angle=70, trapezium right angle=110, minimum width=3cm, minimum height=1cm, text centered, draw=black, fill=blue!30]
\tikzstyle{process} = [rectangle, minimum width=3cm, minimum height=1cm, text centered, text width=3cm, draw=black, fill=orange!30]
\tikzstyle{decision} = [diamond, minimum width=3cm, minimum height=1cm, text centered, draw=black, fill=green!30]
\tikzstyle{arrow} = [thick,->,>=stealth]


\begin{figure}[H]
    \centering
    \begin{tikzpicture}[node distance=2cm]
        % 第一步画结点
        \node (start) [startstop] {开始};
        \node (in1) [io, below of=start] {Input};
        \node (pro1) [process, below of=in1] {Process 1};
        \node (dec1) [decision, below of=pro1, yshift=-0.5cm] {Decision 1};
        \node (pro2a) [process, below of=dec1, yshift=-1.5cm] {Process 2a long long long text long long long text};
        \node (pro2b) [process, right of=dec1, xshift=2cm] {Process 2b};
        \node (out1) [io, below of=pro2a, yshift=-1.5cm] {Output};
        \node (stop) [startstop, below of=out1] {Stop};

        % 第二步画线
        \draw [arrow] (start) -- (in1);
        \draw [arrow] (in1) -- (pro1);
        \draw [arrow] (pro1) -- (dec1);
        \draw [arrow] (dec1) -- node[anchor=east]{yes} (pro2a);
        \draw [arrow] (dec1) -- node[anchor=south]{no} (pro2b);
        \draw [arrow] (pro2b) |- (pro1); %折线画法
        \draw [arrow] (pro2a) -- (out1);
        \draw [arrow] (out1) -- (stop);
    \end{tikzpicture}
    \caption{算法流程图}
\end{figure}

\section{结语}


\end{document}